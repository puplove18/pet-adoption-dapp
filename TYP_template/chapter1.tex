\chapter{Introduction} \label{cha:intro}


Short passage (paragraph or so) telling us what the chapter is about,
and setting the scene. For example,

This chapter provides a brief statement of the overall aim of the
project. It gives an overview of the problem being addressed by the
project. It describe aspects of the background such as why the project
is worth doing, how the project may be useful or helpful for
others. The chapter ends twith a brief chapter by chapter overview of
the rest of the report

\section{First Section} \label{sec:intro:first}

First section of your introduction. We would be surprised if there is
no citation to some literature, e.g.,
\cite{lulcs,PapacchiniCaminatiHu23}.



\subsection{Subsection Example} \label{sec:intro:a_subsection}

Just an example on how to number subsections. Normally you do not need
a nesting deeper than this (i.e., subsubsections).


\section{Signposting} \label{sec:intro:signposting}
Remember to signpost the reader at the end of the introduction. For
example, a very simple example on the use of tables and figures is
given in Chapter~\ref{cha:chapter2}.
